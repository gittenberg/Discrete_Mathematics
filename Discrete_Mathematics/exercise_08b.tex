% Koma class
\documentclass[a4paper, oneside]{scrartcl}   

\usepackage{a4wide}

%------------------
% language = english
\usepackage[english, german]{babel}	% Umlaute mit \"u
\usepackage[latin1]{inputenc}
\usepackage{enumitem}

% margins + Kopf- und Fu�zeilen
\usepackage[left = 2.5cm, right = 2.5cm, top = 2cm, bottom = 3cm]{geometry}
\usepackage{scrpage2} 
\pagestyle{scrheadings}
\clearscrheadfoot
\rehead{\headmark}
\lehead{\pagemark}
\lohead{\headmark}
\rohead{\pagemark} 

% math
\usepackage{amssymb}
\usepackage{amsmath}

% figures
\usepackage{tikz}
\usepackage{graphicx}

% section-Zaehler wird neu gesetzt:
\setcounter{section}{8}
%------------------
\author{Sascha Meiers, Martin Seeger}
\title{Exercise 8, Discrete Mathematics for Bioinformatics}
\date{Winter term 2011/2012}


\begin{document}
\maketitle

%---------------------------------------------------------------------------------------------------

\subsection{Bases and Basic Solutions}

\renewcommand{\labelenumi}{\alph{enumi})}
\begin{enumerate}
\item 
\begin{equation}
x_1+x_2 \leq 4,
\end{equation}
\begin{equation}\label{spurious}
x_1 \leq 4,
\end{equation}
\begin{equation}
-x_1 \leq 0,
\end{equation}
\begin{equation}
x_2 \leq 2,
\end{equation}
\begin{equation}
-x_2 \leq 0.
\end{equation}
We note that (\ref{spurious}) is dependent on the other inequalities and can be
left out. What remains is
\[
Ax = 
\left( \begin{array}{r@{\quad}r@{\quad}r}
1 & 1\\
-1 & 0\\
0 & 1\\
0 & -1
\end{array}\right)
\begin{pmatrix}
x_1 \\ x_2
\end{pmatrix} \leq b = 
\begin{pmatrix}
4 \\ 0 \\ 2 \\ 0
\end{pmatrix}.
\]
\item Let $M = \{1, 2, 3, 4\}$. We look for
bases, i.e. cardinality 2 subsets $I \subset M$ for which $A_{I*}$ is regular.
The bases are
\[
\{1,2\},
\{1,3\},
\{1,4\},
\{2,3\},
\{2,4\}.
\]
The corresponding basic solutions are $A_{I*}^{-1}b_I$, i.e.
\begin{equation}\label{unfeasible}
\left( \begin{array}{rr}
1 & 1\\
-1 & 0
\end{array}\right)^{-1}
\begin{pmatrix}
4 \\ 0
\end{pmatrix} =
\left( \begin{array}{rr}
0 & -1\\
1 & 1
\end{array}\right)
\begin{pmatrix}
4 \\ 0
\end{pmatrix} =
\begin{pmatrix}
0 \\ 4
\end{pmatrix},
\end{equation}

\begin{equation}\label{feasible1}
\left( \begin{array}{rr}
1 & 1\\
0 & 1
\end{array}\right)^{-1}
\begin{pmatrix}
4 \\ 2
\end{pmatrix} =
\left( \begin{array}{rr}
1 & -1\\
0 & 1
\end{array}\right)
\begin{pmatrix}
4 \\ 2
\end{pmatrix} =
\begin{pmatrix}
2 \\ 2
\end{pmatrix},
\end{equation}

\begin{equation}
\left( \begin{array}{rr}
1 & 1\\
0 & -1
\end{array}\right)^{-1}
\begin{pmatrix}
4 \\ 0
\end{pmatrix} =
\left( \begin{array}{rr}
1 & 1\\
0 & -1
\end{array}\right)
\begin{pmatrix}
4 \\ 0
\end{pmatrix} =
\begin{pmatrix}
4 \\ 0
\end{pmatrix},
\end{equation}

\begin{equation}
\left( \begin{array}{rr}
-1 & 0\\
0 & 1
\end{array}\right)^{-1}
\begin{pmatrix}
0 \\ 2
\end{pmatrix} =
\left( \begin{array}{rr}
-1 & 0\\
0 & 1
\end{array}\right)
\begin{pmatrix}
0 \\ 2
\end{pmatrix} =
\begin{pmatrix}
0 \\ 2
\end{pmatrix},
\end{equation}

\begin{equation}\label{feasible4}
\left( \begin{array}{rr}
-1 & 0\\
0 & -1
\end{array}\right)^{-1}
\begin{pmatrix}
0 \\ 0
\end{pmatrix} =
\left( \begin{array}{rr}
-1 & 0\\
0 & -1
\end{array}\right)
\begin{pmatrix}
0 \\ 0
\end{pmatrix} =
\begin{pmatrix}
0 \\ 0
\end{pmatrix}.
\end{equation}

\item Feasible means $Ax \leq b$. This is the case for
(\ref{feasible1})--(\ref{feasible4}) but not for (\ref{unfeasible}), since there
$4 = x_2 > 2$.
\item 
$I = \{1,3\}$: $x = (2,2)$,\\
$I = \{1,4\}$: $x = (4,0)$,\\
$I = \{2,3\}$: $x = (0,2)$,\\
$I = \{2,4\}$: $x = (0,0)$.
\end{enumerate}

\subsection{Simplex Algorithm}

\begin{enumerate}
\item Let $x_1 =$ volume of coke, and $x_2 =$ volume of beer. Then we have
\[
\mathrm{profit} = u^T x = x_1 + 2 x_2 = \mathrm{max!},
\]
\[
\mathrm{weight} = x_1 + 1.5 x_2 \leq 150,
\]
\[
\mathrm{beer\ availability} = x_2 \leq 35,
\]
\[
\mathrm{maximum\ alcoholic\ volume} = (x_1+x_2)\cdot 2/3 \geq x_2
\Longleftrightarrow  -2 x_1 + x_2 \leq 0.
\]
$A$ and $b$ can simply be read off.
\item Simplex algorithm: see handwritten sheet.
\end{enumerate}

\subsection{Duality}

x

\end{document}
