% Koma class
\documentclass[a4paper, oneside]{scrartcl}   

\usepackage{a4wide}

%------------------
% language = english
\usepackage[english, german]{babel}	% Umlaute mit \"u
\usepackage[latin1]{inputenc}

% margins + Kopf- und Fu�zeilen
\usepackage[left = 2.5cm, right = 2.5cm, top = 2cm, bottom = 3cm]{geometry}
\usepackage{scrpage2} 
\pagestyle{scrheadings}
\clearscrheadfoot
\rehead{\headmark}
\lehead{\pagemark}
\lohead{\headmark}
\rohead{\pagemark} 


% math
\usepackage{amssymb}
\usepackage{amsmath}

% figures
\usepackage{tikz}
\usepackage{graphicx}


% section-Zaehler wird neu gesetzt:
\setcounter{section}{3}
%------------------
\author{Sascha Meiers, Martin Seeger}
\title{Exercise 3, Discrete Mathematics for Bioinformatics}
\date{Winter term 2011/2012}


\begin{document}
\maketitle

%---------------------------------------------------------------------------------------------------

\subsection{Skip lists}

\noindent a) Expected value of $h$: we use the notation from the script:
$x\in S$, $h(x) =$ number of sets $S_i$ containing $x$, $h = 1+\max\{h(x):
x\in S\}$. 

For $k\geq 1$, we have $P(h(x) \geq k) = p^{k-1}$ and therefore 
\[
P(h \geq k+1) = nP(h(x) \geq k) = np^{k-1}.
\]
This estimate does not make sense for $k < 1 + \log_{1/p} n=1 - \log_{p} n$.
For those values of $k$ we can use the trivial upper bound $P(h \geq k+1) \leq 
1$. Then $E(h)$ equals:
\begin{eqnarray}
\sum_{k=1}^\infty P(h \geq k+1) &=& \sum_{k=1}^{\lceil - \log_{p} n \rceil} P(h
\geq k+1)+\sum_{k=1+\lceil - \log_{p} n \rceil}^{\infty} P(h \geq k+1) 
\leq \nonumber \\
&\leq& 1+\lceil - \log_{p} n \rceil + \sum_{k=1+\lceil - \log_{p} n
\rceil}^{\infty} np^{k-1}=.
\nonumber 
\end{eqnarray}

\noindent b) Expected value of search time:\\

\noindent c) Expected value of space consumption:\\


\subsection{``Sparse'' skip list}

\noindent a) x\\

\subsection{Skip lists}

\noindent a) x\\


\subsection{Independencies}

We have
\[
E(X_1) = \frac{1}{9}(1+1+2+2+3+3+1+2+3) = 2,
\]
\[
E(X_2) = \frac{1}{9}(2+3+1+3+1+2+1+2+3) = 2,
\]
\[
E(X_3) = \frac{1}{9}(3+2+3+1+2+1+1+2+3) = 2.
\]

\noindent i) x\\

\noindent ii) x\\

\noindent iii) x\\

\noindent iv) x\\

\noindent v) x\\

\noindent vi) $N=X_2$, $E(N)=2$. Therefore,
\[
\sum_{i=1}^{E(N)} E(X_i) = E(X_1) + E(X_2) = 4.
\]
On the other hand,
\begin{eqnarray}
E\left(\sum_{i=1}^{N} X_i\right)&=& 
P(N=1)E\left(\sum_{i=1}^{1} X_i\middle| N=1\right)+
P(N=2)E\left(\sum_{i=1}^{2} X_i\middle| N=2\right)+
\nonumber \\
&+&P(N=3)E\left(\sum_{i=1}^{3} X_i\middle| N=3\right)=
\frac{2}{3}+
\frac{2+2}{3}+
\frac{2+2+3}{3}=\frac{13}{3}.
\nonumber 
\end{eqnarray}

\end{document}
