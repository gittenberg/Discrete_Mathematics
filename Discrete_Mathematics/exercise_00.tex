% test
% math1_1.tex
\documentclass[a4paper]{article}
\usepackage{a4wide}
\usepackage{german}
\usepackage[latin1]{inputenc}
%\usepackage{uebungen}
%------------------
% section-Zaehler wird neu gesetzt:
\setcounter{section}{0}
%------------------
\author{Sascha Meiers, Martin Seeger}
\title{Exercise 0, Discrete Mathematics for Bioinformatics}
\date{Winter term 2011/2012}

\begin{document}
\maketitle

%\section{Title}

\subsection{Background Reading I}

The six (not seven) statements: 

\begin{enumerate}
  \item {\it Conceptualizing, not programming.} Agree.
  \item {\it Fundamental, not rote skill.} Disagree.
  \item {\it A way that humans, not computers, think.} Disagree.
  \item {\it Complements and combines mathematical and engineering thinking.}
  Agree.
  \item {\it Ideas, not artifacts.} Agree.
  \item {\it For everyone, everywhere.} Disagree.
\end{enumerate}

{\bf Statement of strongest agreement:} {\it Ideas, not artifacts.}

Jeanette Wing states that computational concepts will be ``present everywhere and
touch our lives all the time'', besides also the ``software and hardware
artifacts'' that represent them. We would like to argue that this time has
already come by means of two prominent examples:
\begin{itemize}
  \item Social networks are an area in which graph theory has invaded human
  thinking (and probably vice versa). Thanks to the ubiquity of social networks,
  and their mobile availability, participants have stopped thinking about the
  hardware or software on which Facebook or other social applications are
  running. However, they do (usually unknowingly) think about graph 
  theoretical concepts such as connectivity, distances etc. which was certainly less common before the rise
  of social networks.
  \item Similarly, navigation and geolocation systems help manage our daily
  lives and make (some of) us think about a whole other range of problems, e.g.
  how to calculate shortest paths.
\end{itemize}

One could add further examples, such as optimization and search
problems which have recently arrived in a large part of the broad population and
have begun to shape their thinking.
\\

{\bf Statement of strongest disagreement:} {\it For everyone, everywhere.}

Having argued in the previous paragraph that computational concepts are
ubiquitous and influence our lives already now, we would now claim that this
does probably not happen in the way the author envisages when she says
it might become ``so integral to human
endeavors'' that ``it disappears as an explicit philosophy''. It is wishful
thinking to believe that more than a minority of the population would
significantly change their attitude towards computational thinking which she
likens to mathematical and engineering thinking in one of the earlier
statements.

Realistically, the majority of the population will not change their style of
thinking and their attitude towards mathematics or computer science by more
exposure to social networks, for example, or to other manifestations of
computing techniques. For this to happen, profound and significant changes would have to
occur in early education and even then ``computational thinking for everyone''
will remain an illusion. People will stay the way they are, pretty much, no
matter how much computation is happening around them.

\subsection{Background Reading II}

If $n=0$ is permitted, then it is the smallest such value since $0\leq 2^0$.
Otherwise, the inequality
\begin{equation}
100 n^2 \leq 2^n
\end{equation}
needs to be solved for minimal $n>0$. One finds up to $n=14$, at which
\begin{equation}
100 \cdot 14^2 = 19600 > 2^{14} = 16384,
\end{equation}
whereas for $n=15$
\begin{equation}
100 \cdot 15^2 = 22500 \leq 2^{15}= 32768.
\end{equation}
Hence, 15 is the smallest value such that the quadratic algorithm runs faster.

\subsection{Background Reading III}

Example of a Monte Carlo algorithm: numerical integration, in particular in high
dimensions.

\noindent Example of a Las Vegas algorithm: randomized Quick Sort.

\end{document}