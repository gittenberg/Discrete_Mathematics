% Koma class
\documentclass[a4paper, oneside]{scrartcl}   

\usepackage{a4wide}

%------------------
% language = english
\usepackage[english, german]{babel}	% Umlaute mit \"u
\usepackage[latin1]{inputenc}

% margins + Kopf- und Fu�zeilen
\usepackage[left = 2.5cm, right = 2.5cm, top = 2cm, bottom = 3cm]{geometry}
\usepackage{scrpage2} 
\pagestyle{scrheadings}
\clearscrheadfoot
\rehead{\headmark}
\lehead{\pagemark}
\lohead{\headmark}
\rohead{\pagemark} 


% math
\usepackage{amssymb}
\usepackage{amsmath}

% figures
\usepackage{tikz}
\usepackage{graphicx}


% section-Zaehler wird neu gesetzt:
\setcounter{section}{2}
%------------------
\author{Sascha Meiers, Martin Seeger}
\title{Exercise 2, Discrete Mathematics for Bioinformatics}
\date{Winter term 2011/2012}


\begin{document}
\maketitle

%---------------------------------------------------------------------------------------------------

\subsection{Modulo Arithmetic}

\noindent a) We show that $\langle a \rangle \subset \langle d \rangle$.

Since $d = \mathrm{gcd}(a, n)$, there is a $k \in \mathbb{N}$ such that $a = kd$.
Hence, if $v \in \langle a \rangle$, i.e. $v = ai \mod n$, then $v = dki \mod n$ which implies that $v \in \langle d \rangle$.
\\

\noindent b) We show that $\langle a \rangle \supset \langle d \rangle$.

Any element $v$ of $\langle d \rangle$ can be written as $v = di\mod n$ (*).
On the other hand, $v \in \langle a \rangle$ iff $v = aj\mod n$.

We now use Bezout's lemma to find $x$, $y$, such that $ax + ny = d$. This is inserted into (*) to yield
\[
v = di\mod n = (ax + ny)i\mod n = axi\mod n.
\]
In other words, $v \in \langle a \rangle$. $\square$
\\

\subsection{Hashing}

Let $x,y$ be character strings both of length $n$. Now we can interprete their characters 
as numbers in radix $2^p$, leading to a hash function 
\[h(x) = \sum\limits_{i=0}^n x_i 2^{p\cdot i} \text{ mod } 2^p-1\]
If $y$ is nothing else than a permutation of the characters in $x$, 
then especially their sum of the digits is equal, i.e.
\[ \sum\limits_{i=0}^n x_i = \sum\limits_{i=0}^n y_i  \]
Proof: $h(x) = h(y)$
\begin{align}
h(x) &= \sum\limits_{i=0}^n x_i 2^{p\cdot i} \text{ mod } 2^p-1  \\
     &= \sum\limits_{i=0}^n \left( x_i 2^{p\cdot i} \text{ mod } 2^p-1 \right) \text{ mod } 2^p-1  \\
     &= \sum\limits_{i=0}^n \left( x_i \text{ mod } 2^p-1 \right) \left( \underbrace{2^p \text{ mod } 2^p-1}_1 \right)^i \text{ mod } 2^p-1 \\
     &= \sum\limits_{i=0}^n x_i \text{ mod } 2^p-1  \\
     &= \sum\limits_{i=0}^n y_i \text{ mod } 2^p-1  \\
     &= h(y)
\end{align}


\subsection{Hashing}

x

\subsection{Expected value}



\end{document}
