% Koma class
\documentclass[a4paper, oneside]{scrartcl}   

\usepackage{a4wide}

%------------------
% language = english
\usepackage[english, german]{babel}	% Umlaute mit \"u
\usepackage[latin1]{inputenc}
\usepackage{enumitem}

% margins + Kopf- und Fu�zeilen
\usepackage[left = 2.5cm, right = 2.5cm, top = 2cm, bottom = 3cm]{geometry}
\usepackage{scrpage2} 
\pagestyle{scrheadings}
\clearscrheadfoot
\rehead{\headmark}
\lehead{\pagemark}
\lohead{\headmark}
\rohead{\pagemark} 

% math
\usepackage{amssymb}
\usepackage{amsmath}

% figures
\usepackage{tikz}
\usepackage{graphicx}

% section-Zaehler wird neu gesetzt:
\setcounter{section}{8}
%------------------
\author{Sascha Meiers, Martin Seeger}
\title{Exercise 8, Discrete Mathematics for Bioinformatics}
\date{Winter term 2011/2012}


\begin{document}
\maketitle

%---------------------------------------------------------------------------------------------------

\subsection{Linear Optimization}

\begin{equation}
2x_1+3x_2 = \min \Leftrightarrow -2x_1-3x_2 = \max .
\end{equation}
\begin{equation}\label{1}
3x_1+6x_2\leq 7 \Leftrightarrow 3x_1+6x_2+x_3=7,\ x_3\geq 0.
\end{equation}
\[
x_1\  \mathrm{free} \Leftrightarrow x_1 = x_4-x_5,\ x_4\geq 0,\ x_5\geq 0.
\]
Insert the previous equation into \eqref{1}:
\[
6x_2+x_3+3x_4-3x_5=7,\ x_3\geq 0,\ x_4\geq 0,\ x_5\geq 0.
\]
Finally,
\[
2x_1 + 2x_2 = 5 \Leftrightarrow 2x_2 + 2x_4 - 2x_5= 5, 
\]
\[
-2x_1-3x_2 = -3x_2-2x_4+2x_5 = \max .
\]
Summarizing,
\[
\max \left\{(-3,0,-2,2) 
\left(
 \begin{smallmatrix}
  x_2\\x_3\\x_4\\x_5
 \end{smallmatrix}
\right)
\middle| 
\begin{pmatrix}
 6 & 1 & 3 & -3\\
 2 & 0 & 2 & -2
\end{pmatrix}
\left(
 \begin{smallmatrix}
  x_2\\x_3\\x_4\\x_5
 \end{smallmatrix}\right) =
 \begin{pmatrix}
  7\\5
 \end{pmatrix},
\left(
 \begin{smallmatrix}
  x_2\\x_3\\x_4\\x_5
 \end{smallmatrix} \right)
 \geq 0
\right\}.
\]

\subsection{Linear Optimization}

x

\subsection{Linear Optimization}

\begin{enumerate}[label={\alph*)}]
  \item For $c^T = (-1, -1)$ we have to maximize $-x_1-x_2$, i.e. minimize
  $x_1+x_2$.\\
  In the feasible region, this is the case for $(x_1, x_2) = (0, 0)$.
  \item For $c^T = (0, -1)$ we have to maximize $-x_2$, i.e. minimize
  $x_2$.\\ 
  In the feasible region, this is the case for $(x_1, x_2) = (\lambda, 0)$,
  $\lambda \in [0, 1]$.
  \item For $c^T = (-1, 0)$ we have to maximize $-x_1$, i.e. minimize
  $x_1$.\\ 
  In the feasible region, this is the case for $(x_1, x_2) = (0, \lambda)$,
  $\lambda \in [0, \infty [$.
  \item For $c^T = (1, 1)$ we have to maximize $x_1+x_2$.\\
  Since e.g. the half-line $x_1-x_2=0$, $x_1, x_2 \geq 0$ lies in the feasible
  region, and as the target function increases strictly along this half-line,
  there is no optimal solution.
\end{enumerate} 
If the following constraint is added, the problem becomes infeasible:
\[
x_1-x_2 \geq 2.
\]

\subsection{Profit Optimization}

x

\end{document}
