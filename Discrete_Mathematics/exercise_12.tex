% Koma class
\documentclass[a4paper, oneside]{scrartcl}   

\usepackage{a4wide}

%------------------
% language = english
\usepackage[english, german]{babel}	% Umlaute mit \"u
\usepackage[latin1]{inputenc}
\usepackage{enumitem}

% margins + Kopf- und Fusszeilen
\usepackage[left = 2.5cm, right = 2.5cm, top = 2cm, bottom = 3cm]{geometry}
\usepackage{scrpage2} 
\pagestyle{scrheadings}
\clearscrheadfoot
\rehead{\headmark}
\lehead{\pagemark}
\lohead{\headmark}
\rohead{\pagemark} 

% math
\usepackage{amssymb}
\usepackage{amsmath}

% figures
\usepackage{tikz}
\usepackage{graphicx}

% section-Zaehler wird neu gesetzt:
\setcounter{section}{12}
%------------------
\author{Sascha Meiers, Martin Seeger}
\title{Exercise 12, Discrete Mathematics for Bioinformatics}
\date{Winter term 2011/2012}


\begin{document}
\maketitle

%---------------------------------------------------------------------------------------------------

\subsection{Inverse Queens Problem}

\renewcommand{\labelenumi}{\alph{enumi})}
\begin{enumerate}
  \item 
  Variables
    \[x_i \in \{1,...,n\} \quad \text{ for } 1 \leq i \leq n \]
  Constraints
    \[ x_i = x_j  \vee |x_i -x_j| = |i-j| \quad  \forall i\neq j \]
    
  \item
  Solve for $n=4$ and $D_1 = \{2\}$.
  
  \paragraph{Forward checking:} $D_1 = D_2 = D_3 = D_4 = \{1,2,3,4\}$
  \begin{itemize} 
  \renewcommand{\labelitemi}{$\bullet$}
  \renewcommand{\labelitemii}{$\bullet$}
  \renewcommand{\labelitemiii}{$\bullet$}
  \renewcommand{\labelitemiv}{$\bullet$}
        \item $x_1 = 2 \Rightarrow D_2 = \{1,2,3\}, D_3 = \{2,4\}, D_4 = \{2\}$
        \begin{itemize}
            \item $x_2 = 1 \Rightarrow D_3 = \{2\}, D_4 = \{\}$ ... dead end.
            \item $x_2 = 2 \Rightarrow D_3 = \{2\}, D_4 = \{2\}$
            \begin{itemize}
                \item Solution found
            \end{itemize}
        \end{itemize}
  \end{itemize}
  

  \paragraph{Patial lookahead:} $D_1 = D_2 = D_3 = D_4 = \{1,2,3,4\}$
  \begin{itemize} 
  \renewcommand{\labelitemi}{$\bullet$}
  \renewcommand{\labelitemii}{$\bullet$}
  \renewcommand{\labelitemiii}{$\bullet$}
  \renewcommand{\labelitemiv}{$\bullet$}
        \item $x_1 = 2 \Rightarrow D_2 = \{2\}$ because values $1$ or $3$ are not arc consistent with $x_4$. 
              $D_3 = \{2\}$ because value $4$ is not arc consistent with $x_4$. $D_4 = \{2\}$.
        \begin{itemize}
            \item Solution found
        \end{itemize}
  \end{itemize}
    

\end{enumerate}

\subsection{Task Scheduling}

\subsection{Bin Packing}

\subsection{IP}

%Additionally to the variables $x_1, ..., x_n \in \{0,...,m\}$ we introduce decision variables $d_21,d_31,d_...,d_n \in \{0,1\}$.
Each constraint of the form $|x_i - x_j| \geq 2$ can be rewritten as
\[x_i - x_j \geq 2 \vee x_j - x_i \geq 2\]
We can express the logical \emph{or} by adding a 
new variable $d_{ij}$ (decision variable):
\[ \begin{array} {ccc}
x_i-x_j \geq 2 & \vee & x_j-x_i \geq 2 \\
\Leftrightarrow x_i - x_j \geq d_{ij}(-2-m) +2 &\wedge& x_j - x_i \geq (1-d_{ij})(-2-m) +2 \\
\Leftrightarrow x_i - x_j + (2+m)d_{ij} \geq 2 &\wedge& x_j - x_i - (2+m)d_{ij}\geq  -m \\
\end{array} \]
The decision variable decides which constraint must be satisfied: 
\[ \begin{array} {cccc}
\text{if } d_{ij} = 0 \qquad & x_i - x_j \geq 2 &\wedge& x_j - x_i \geq -m \\
\text{if } d_{ij} = 1 \qquad & x_i - x_j \geq -m &\wedge& x_j - x_i + \geq  2 \\
\end{array} \]
Note that these equivalencies only hold since $x_i-x_j \geq 2$ and $x_j-x_i \geq 2$ cannot be true at the same time.

In order to model these inequalities $\forall i\neq j$, we have to apply this technique to $n^2  -n$ constraints, 
thus introducing $(n^2-n)$ new binary variables.


\end{document}
