% math1_1.tex
\documentclass[a4paper]{article}
\usepackage{a4wide}
\usepackage{german}
\usepackage{amssymb}
\usepackage[latin1]{inputenc}
%\usepackage{uebungen}
%------------------
% section-Zaehler wird neu gesetzt:
\setcounter{section}{1}
%------------------
\author{Sascha Meiers, Martin Seeger}
\title{Exercise 1, Discrete Mathematics for Bioinformatics}
\date{Winter term 2011/2012}

\begin{document}
\maketitle

%\section{Title}

\subsection{MST Approximation}

x

\subsection{Landau Symbols}

\renewcommand{\labelenumi}{\alph{enumi})}
\begin{enumerate}
\item Let $k, l \in \mathbb{Z}$, $k > l$. $f = o(g)$ holds iff 
\begin{equation}
\lim_{n\rightarrow \infty} \left|\frac{f(n)}{g(n)}\right| = 0. 
\end{equation}
In our case, 
\begin{equation}
\lim_{n\rightarrow \infty} \left|\frac{f(n)}{g(n)}\right| = \lim_{n\rightarrow
\infty} \left|\frac{n^l}{n^k}\right| = \lim_{n\rightarrow
\infty} \left|\frac{1}{n^{k-l}}\right| = 0,
\end{equation}
whence it follows that $n^l=o(n^k)$.$\square$

\item Let $k, l \in \mathbb{N}$, $k > l$. In general, $f=\Theta(g)$ iff
$f=O(g)$ and $g=O(f)$. We use the definition $f=O(g)$ iff
\begin{equation}
0 \leq \limsup_{n\rightarrow \infty} \left|\frac{f(n)}{g(n)}\right| < \infty. 
\end{equation}
In our case, 
\begin{equation}
\limsup_{n\rightarrow \infty} \left|\frac{n^k+n^l}{n^k}\right| =
\limsup_{n\rightarrow \infty} \left|1 + \frac{1}{n^{k-l}}\right| = 1,
\end{equation}
and
\begin{equation}
\limsup_{n\rightarrow \infty} \left|\frac{n^k}{n^k+n^l}\right| =
\limsup_{n\rightarrow \infty} \left|1 - \frac{n^l}{n^k+n^l}\right| =
\limsup_{n\rightarrow \infty} \left|1 - \frac{1}{n^{k-l}+1}\right| =
1.\square
\end{equation}
 
\item Counterexample: $f(n)=2^{cn}$ with $c>1$ is clearly $2^{O(n)}$. However, 
\begin{equation}
\limsup_{n\rightarrow \infty} \left|\frac{2^{cn}}{2^{n}}\right| =
\limsup_{n\rightarrow \infty} 2^{(c-1)n} = \infty,
\end{equation}
hence $f \neq O(2^n)$.$\square$
\end{enumerate}


\subsection{Amortized Analysis}

x

\subsection{Analysis of SELECTION algorithm}

x

\end{document}